%Identifying parts or objects in images is a effortless task for humans but a complex assignment for computers. 
Boundary detection and region segmentation have been extensively studied for over 50 years, with several approaches.
Latterly, machine learning, specially convolutional neural networks (CNN), techniques have proven quite effective in solving these problems.
%Among them, those that use convolutional neural networks (CNN) stand out. 
%The methods that exist today, despite the greater precision compared to some years ago, are still evolving and can be enhanced.
A recent improvement is to combine features generated in multiple network layers. 
Due to their architecture, CNNs produce different information along their layers, on a multiple scale, which, when combined, contribute to the final result with their own characteristics.
Some works, aiming to increase performance, decided to train individual convolutional blocks to force this behavior, producing similar maps in muliple scales.
This method, however, increases the cost and time of training, once it does not take advantage of information previously generated by correlated problems.
%This relationship between problems enables the production of good results with a low number of training epochs, making these solutions suitable to proof of concepts, conditions with high cost or events with close deadlines.
This work seeks to propose and evaluated trivial techniques to combine features resulting from different layers of CNNs, without generate similar multiple outputs, to produce boundary detection and region segmentation. %, using previous results from well-know object detection / classification networks.
It evaluates the influence of the number of intermediate results extracted from the network (side-outputs) and what trivial operations, such as average, maximum and sum, can be used in those tasks.
Also evaluated how to combine multiple multiple ground truths annotations, presented in some data sets, into a single reference map, to increase convergence.
%The creation of simple or even trivial methods favors the use in different scenarios, once there is no attempt to solve the uniqueness of each problem.
The creation of simple or even trivial methods favors the use in different scenarios, becoming a generalist method.
The networks developed here were tested for region segmentation and edge detection tasks, with performance comparable to the literature, despite its simplicity.
In the edge detection task, our best results reached 0.780 ODS on the well-known BSDS500 data set, at \myFPS FPS.
